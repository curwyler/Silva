\documentclass[9pt]{article}
\author{Cedar Urwyler}

\usepackage[utf8]{inputenc}
\usepackage[ngerman]{babel}
\usepackage{graphicx}
\usepackage[width=16cm, top=4cm, headsep=1.5cm]{geometry}

\usepackage{multicol}
\setlength{\columnsep}{25pt} %Spaltenabstand

\usepackage{pdflscape}

\usepackage{xcolor}
\usepackage{framed}
\colorlet{shadecolor}{brown!20} %Boxfarbe


%Redefining stuff
\makeatletter

\renewcommand\thesection{}
\renewcommand\thesubsection{}
\renewcommand\thesubsubsection{}

\makeatother

\pagestyle{myheadings}

\title{Das Grüne Auge:\\DSA-Regeln für Silva}



\begin{document}
\maketitle
\tableofcontents

\section{Kreaturenregeln}
\begin{multicols}{2}

\begin{shaded*}
	\subsubsection{Gnoll-Krieger}
	\textbf{Grösse:} bis 2m\\
	\textbf{Gewicht:} bis 90 kg\\
	\begin{tabular}{cccc}
		&&&\\
		\textbf{MU} 13 & \textbf{KL} 9 & \textbf{IN} 12 & \textbf{CH} 9 \\ 
		\textbf{FF} 12 & \textbf{GE} 15 & \textbf{KO} 13 & \textbf{KK} 13 \\
		&&&\\
	\end{tabular}
	\\
	\textbf{LeP} 31 \textbf{AsP} -	\textbf{INI} 11+1W6 \\
	\textbf{RS} 1 	\textbf{SK} 0	\textbf{ZK} 2 		\textbf{GS} 10 \\
	\textbf{Ausweichen} 7 \\
	\textbf{Biss} \textbf{AT} 11 \textbf{PA} - \textbf{TP} 1W6+2\\
	\textbf{Morgenstern:} \textbf{AT} 13 \textbf{PA} - \textbf{TP} 1W6+5\\
	\textbf{Kurzbogen:} \textbf{AT} 13 \textbf{LZ} 1 \textbf{TP} 1W6+4 \textbf{RW} 10/50/80\\
	\textbf{Aktionen:} 1 \\
	\textbf{Vorteil:} Dunkelsicht I, Herausragender Sinn (Geruch) \\
	\textbf{Nachteil:} Blutrausch\\
	\textbf{Talente:}
	Körperbeherrschung 5,
	Verbergen 7,
	Sinnesschärfe 8
	\\
	\textbf{Anzahl:} 2W6 (Jagdrudl) \\
	\textbf{Größenkategorie:} normal \\
	\textbf{Typus:} Gnoll, humanoid \\
	
	
	%			
	%			\textbf{AP-Wert:} 16 Abenteuerpunkte
	%			\\
	%			\textbf{Lebensenergie-Grundwert:} 5
	%			\\
	%			\textbf{Seelenkraft-Grundwert:} –6
	%			\\
	%			\textbf{Zähigkeit-Grundwert:} –4
	%			\\
	%			\textbf{Geschwindigkeit-Grundwert:} 10
	%			\\
	%			\textbf{Eigenschaftsänderungen:} GE und KK +1; KL oder CH –2\\
	%			\textbf{Dringend empfohlene Vor- und Nachteile:} Folgende Vor- und Nachteile zeichnen Gnolle im besonderen Maße aus. Diese Vor- und Nachteile sollten gewählt werden: Blutrausch, Dunkelsicht I, Herausragender Sinn (Geruch), Schlechte Eigenschaften (Jähzorn)
	%			\\
	%			\textbf{Übliche Kulturen:} Barbaren, Pazifisten
	%			\\
	%			\textbf{Typische Vorteile:} Flink, Hitzeresistenz, Richtungssinn, Zäher Hund
	%			\\
	%			\textbf{Typische Nachteile:} Angst vor Feuer, Hässlich I-II, Zauberanfällig I-II
	%			\\
	%			\textbf{Untypische Vorteile:} keine
	%			\\
	%			\textbf{Untypische Nachteile:} Behäbig, Fettleibig, Hitzeempfindlich, Nachtblind
\end{shaded*}
\begin{shaded*}
	\subsubsection{Gnome als Spielerrasse}
	\textbf{AP-Wert:} 22 Abenteuerpunkte
	\\
	\textbf{Lebensenergie-Grundwert:} 5
	\\
	\textbf{Seelenkraft-Grundwert:} –4
	\\
	\textbf{Zähigkeit-Grundwert:} –5
	\\
	\textbf{Geschwindigkeit-Grundwert:} 6
	\\
	\textbf{Eigenschaftsänderungen:} KL und IN +1; KO oder KK –2\\
	\textbf{Dringend empfohlene Vor- und Nachteile:} Folgende Vor- und Nachteile zeichnen Gnome im besonderen Maße aus. Diese Vor- und Nachteile sollten gewählt werden	oder es muss mit dem Spielleiter abgestimmt werden, warum darauf verzichtet wird: Dunkelsicht I, Persönlichkeitsschwächen
	\\
	\textbf{Übliche Kulturen:} Hügelgnome, Tiefengnome (Svirfnebli)
	\\
	\textbf{Typische Vorteile:} Begabung, Dunkelsicht II, Fuchssinn, Richtungssinn, Zwergennase
	\\
	\textbf{Typische Nachteile:} keine 
	\\
	\textbf{Untypische Vorteile:} keine
	\\
	\textbf{Untypische Nachteile:} Blutrausch, Jähzorn, Nachtblind
\end{shaded*}
\begin{shaded*}
	\subsubsection{Goblins als Spielerrasse}
	\textbf{AP-Wert:} 0 Abenteuerpunkte
	\\
	\textbf{Lebensenergie-Grundwert:} 5
	\\
	\textbf{Seelenkraft-Grundwert:} –5
	\\
	\textbf{Zähigkeit-Grundwert:} –5
	\\
	\textbf{Geschwindigkeit-Grundwert:} 8
	\\
	\textbf{Eigenschaftsänderungen:} FF und KL +1; CH oder KK –2\\
	\textbf{Dringend empfohlene Vor- und Nachteile:} Folgende Vor- und Nachteile zeichnen Goblins im besonderen Maße aus. Diese Vor- und Nachteile sollten gewählt werden oder es muss mit dem Spielleiter abgestimmt werden, warum darauf verzichtet wird: Dunkelsicht I, Herausragender Sinn (Gehör), Immunität gegen die Fäule
	\\
	\textbf{Typische Vorteile:} Begabung(Abrichten), Flink, Krankheitsresistenz I-II
	\\
\end{shaded*}
\begin{shaded*}
	\subsubsection{Regeln für Oger}
	Benutze die Regeln von ofizieller Seite, ergänzt um Hörner, Klauen und vielleicht eine zusätzliche Aktion für mehrköpfige und/oder mehrarmige Oger. Einige Oger können primitive Magie wirken.
\end{shaded*}
	
	\begin{shaded*}		
		\subsubsection{Squigs}
		\textbf{Grösse:} bis 20 cm\\
		\textbf{Gewicht:} bis 3 kg\\
		\begin{tabular}{cccc}
			&&&\\
			\textbf{MU} 12 & \textbf{KL} 8(t) & \textbf{IN} 10 & \textbf{CH} 8 \\ 
			\textbf{FF} 12 & \textbf{GE} 14 & \textbf{KO} 7 & \textbf{KK} 5 \\
			&&&\\
		\end{tabular}
		\\
		\textbf{LeP} 10 \textbf{AsP} -	\textbf{INI} 13+1W6 \\
		\textbf{RS} 1 	\textbf{SK} 1	\textbf{ZK} -2		\textbf{GS} 7 \\
		\textbf{Verteidigung} 7 \\
		\textbf{Biss: }\textbf{AT} 13 \textbf{TP } 1W6 \textbf{RW }kurz \\
		\textbf{Aktionen:} 1 \\
		\textbf{Sonderfertigkeit:} Festbeissen \\
		\textbf{Talente:} Körperbeherrschung 7, Kraftakt 6, Selbstbeherrschung 4, Sinnesschärfe 7, Einschüchtern 4, Willenskraft 2
		\\
		\textbf{Anzahl:} 2W6 \\
		\textbf{Größenkategorie:} winzig \\
		\textbf{Typus:} Squig, nicht humanoid \\
	\end{shaded*}
\begin{shaded*}		
	\subsubsection{Wichte}
	\textbf{Grösse:} bis 40 cm\\
	\textbf{Gewicht:} bis 7 kg\\
	\begin{tabular}{cccc}
		&&&\\
		\textbf{MU} 8 & \textbf{KL} 8 & \textbf{IN} 15 & \textbf{CH} 15 \\ 
		\textbf{FF} 13 & \textbf{GE} 14 & \textbf{KO} 7 & \textbf{KK} 6 \\
		&&&\\
	\end{tabular}
	\\
	\textbf{LeP} 14 \textbf{AsP} $\infty$	\textbf{INI} 11+1W6 \\
	\textbf{RS} 0 	\textbf{SK} 10	\textbf{ZK} -2 		\textbf{GS} 7 \\
	\textbf{Verteidigung} 7 \\
	\textbf{Aktionen:} 1 \\
	\textbf{Vorteil:} Dunkelsicht I, Herausragender Sinn (Gehör) \\
	\textbf{Talente:} Klettern 10,
	Körperbeherrschung 6,
	Selbstbeherrschung 2,
	Sinnesschärfe 10 (20 bei Nachtschwärmer), Verbergen 10,
	Willenskraft 4
	\\
	\textbf{Zaubertricks:} Alle\\
	\textbf{Zauber:} (Fürchtewicht) Axxeleratus, Visibili; (Nachtschwärmer) Blick in die Gedanken; (Pustelwicht) Hexengalle, Krötensprung;
	\\
	\textbf{Anzahl:} 1 oder 1W6 (Familie) \\
	\textbf{Größenkategorie:} winzig \\
	\textbf{Typus:} Wichtel, humanoid \\
\end{shaded*}

	\begin{itemize}
		\item Wichtel beherrschen alle angegebenen Zauber mit
		einem Fertigkeitswert von 12. Wenn nicht anders
		angegeben, werden die Zauberproben auf die
		Eigenschaftswerte 14/14/14 abgelegt.
		\item Wichtel benötigen immer nur eine Aktion, um einen Zauber zu wirken.
	\end{itemize}
	


\end{multicols}
\section{weitere Regeln}
Falls nicht die Optionalregel 'keine Gegengottheiten' verwendet wird, ist die \textbf{Gegengottheit} der Schattenhexen der Erzdämon Blakharaz.

\subsubsection{Segnungen, Liturgien und Zeremonien} Durch den \textbf{Aspekt Schatten}, haben die Schattenhexen Zugriff auf alle Liturgien und Zeremonien mit Verbreitung Phex(Schatten).

Durch den \textbf{Aspekt Insurrektion} haben sie Zugriff auf: \textit{Wundersame Verständigung, Mondsilberzunge} (Bonus auf Überzeugen und Bekehren statt Handel(Feilschen))

\begin{shaded*}
	\subsubsection{Tradition der Schattenhexen}
	\begin{itemize}
		\item Klarer Verstand: Für die Geweihte sind die Auswirkungen von Verwirrung um eine Stufe gesenkt (z.B. wird Verwirrung Stufe II wie Stufe I behandelt). Ausnahme ist Stufe IV, hier erleidet auch die Schattenhexe alle Einbußen vollständig. 
		\item Die Geweihte erhält eine neue Einsatzmöglichkeit für Selbstbeherrschung (Handlungsfähigkeit bewahren): Wenn eine Schattenhexe durch Zustände handlungsunfähig wird, kann sie trotzdem noch eine Aktion oder Verteidigung pro Kampfrunde ausführen, wenn ihr eine Probe auf Selbstbeherrschung (Handlungsfähigkeit bewahren) gelingt.
		\item Eine Schattenhexe muss sich an den Moralkodex der Schattenhexen halten. Die Wahl des Nachteils ist Voraussetzung, wenn der Spieler eine Schattenhexe spielen will.
		\item Die gefälligen Talente sind: Selbstbeherrschung, Verbergen, Betören, Bekehren \& Überzeugen, Menschenkenntnis, Überreden, Willenskraft, Kriegskunst, Handel, alle Heilkunden.
		\item Leiteigenschaft der Tradition ist Intuition.
		\item[] Kosten: 150 AP
	\end{itemize}
\end{shaded*}
%\subsubsection{Simulacrum}
%Schattenhexen beherrschen die Kunst, einen Schatten zum Leben zu erwecken. Regeltechnisch funktioniert das wie die SF Vertrautenbindung einer zauberkräftigen Hexe. Das sogenannte Simulacrum nimmt dabei eine Tiergestalt an. Zugelassen sind alle Tiere, die auch eine Hexe als Vertrautentier wählen könnte. Es erhält die Grundwerte des entsprechenden Tieres und folgende Eigenschaften:
%
%\begin{itemize}
%	\item Die Werte des Simulacrum sind wie folgt modifiziert +5 LeP, +1 RS, +1 SK, +1 ZK, 15 AsP
%	\item Angriffe mit profanen Waffen verursachen nur den halben Schaden. Die Trefferpunkte werden erst halbiert, dann wird der Rüstungsschutz verrechnet.
%	\item Angriffe mit magischen und geweihten Waffen verursacht regulären Schaden.
%	\item Simulacri sind immun gegen Zauber mit den Merkmalen Einfluss, Heilung, Illusion und Verwandlung.
%	\item Gegen alle anderen auf sie gewirkten Zauber gilt ihre	Seelenkraft (falls sie nicht ohnehin zu einer Erleichterung	führen würde) zusätzlich als Erschwernis (also gegebenenfalls doppelt, falls der Zauber von Natur aus um die Seelenkraft seines Ziels erschwert ist oder zusätzlich, wenn er um die Zähigkeit erschwert	ist). Gegen Zauber, die direkt Schaden verursachen,	gilt die Seelenkraft stattdessen als zusätzlicher Rüstungsschutz.
%	\item Simulacri sind immun gegen Gifte und Krankheiten.
%\end{itemize}
%Simulacri können eine Reihe von Tricks erlernen, die sie unter Aufwendung eigener AsP nutzen können. Dies funtioniert wie bei Vertrautentieren.
%
%Simulacri erhalten AP und können Eigenschaften steigern wie Vertrautentiere von Hexen.


%	\begin{shaded*}
%		\subsubsection{SF: Simulacrum}
%		Die Schattenhexe ist in der Lage, aus ihrem Schatten einen Vertrauten, das Simulacrum, hervorzubringen. Zwar ist dieses Wesen eine eigene Persönlichkeit, andererseits besteht es aus einem Splitter des Schattens der Hexe.
%		Stirbt das Simulacrum, so fährt es zurück in den Schatten der Hexe und kommt vor dem nächsten Neumond nicht wieder hervor. Während dieser Zeit erleidet die Hexe eine Stufe Schmerz. \\
%		Kosten: 20 AP
%	\end{shaded*}
	\begin{shaded}
		\subsubsection{Schiesspulverwaffen}
		\begin{multicols}{2}
			\paragraph{Schiessen} Schiesspulverwaffen werden über die Kampftechnik Armbrüste verwendet.
			\paragraph{Laden} Wenn jemand eine Schiesspulverwaffe lädt, ohne die SF Umgang mit Schwarzfeuerwaffen zu beherrschen, wirft der SL verdeckt 1W3.
			\begin{itemize}
				\item[1-4] Die Waffe funtioniert normal.
				\item[5-6] Wenn jemand versucht, die Waffe abzufeuern, ist es automatisch ein Patzer.
			\end{itemize}
			\paragraph{Falsches Kaliber} Bei der Verwendung einer Kugel, die nicht genau das richtig Kaliber hat, ist das Treffen um 2 erschwert und die Waffe richtet 1 TP weniger an.
			\paragraph{Reichweite*} Mit der Muskete kann bis zum zwei- statt anderthalbfachen über die Reichweite weit hinausgeschossen werden. Solche Schüsse sind immer ungezielt und reichen höchstens dazu aus, um auf eine größere Fläche, über eine Stadtmauer oder ins
			Schlachtgetümmel zu schießen.
		\end{multicols}
		\begin{tabular}{cccccccc}
			\hline \textbf{Waffe} & \textbf{TP} & \textbf{LZ} & \textbf{RW} & \textbf{Munitionstyp} & \textbf{Gewicht} & \textbf{Länge} & \textbf{Preis} \\ 
			Pistole & 1W6+5 & 15 Aktionen & 5/25/40 & Kugeln & 0.75 kg & 30 cm & 180 ST \\
			Muskete & 2W6+8 & 20 Aktionen & 10/50/80* & Kugeln & & & 450 ST \\
			\hline
		\end{tabular}
		\\
	\end{shaded}

	\begin{shaded*}
		\subsubsection{SF: Umgang mit Schwarzpulver}
		Ohne passende Aussbildung können Feuerwaffen nicht zuverlässig geladen und gewartet werden. Mit dieser muss man den Bedienungstest nicht durchführen und kann auch bei feuchtem Wetter Schiesspulver verwenden, wenn die Ausrüstung da ist, es trocken zu halten. Letztes Wort hat der Spielleiter.\\
		Kosten: 10 AP
	\end{shaded*}

\begin{shaded*}
	\subsubsection{Liturgien und Zeremonien erlernen}
	
	Ein Geweihter kann zusätzliche Pilgerwege begehen um neue Liturgien zu erlernen. Unabhängig vom Weg wird er dabei aber in aller Regel nur weitere Liturgien der eigenen Tradition erforschen. Einmal geweiht, können Liturgien und Zeremonien auch von einem Lehrmeister erlernt werden (Der selbst nicht unbedingt geweiht ist, aber doch ein tiefes Verständnis der mystischen Energien und des Wirkens der Heiligen hat). Auch ein Buch kann als Lehrmeister gelten.
\end{shaded*}

\begin{shaded*}
	\subsubsection{Optional: Keine Gegengottheiten}
	Wenn diese Optionalregel verwendet wird, entspricht den DSA-Göttern, die hier Heilige sind, kein Dämon als Gegengottheit. Da dann geweihte Waffen niemals doppelten Schaden anrichten, sollte man auch verfügen, dass geweihte Waffen gegen alle Dämonen pauschal 1W6 Zusatzschaden verursachen.\\
	Dies macht das Spiel etwas weniger komplex.
\end{shaded*}

\begin{shaded*}
	\subsubsection{Mysterienkult um den Feilscher, den Henker \& den Rabenvater}
	Behandle die Heiligen des Mysterienkults einfach wie die Traditionen Phex (Feilscher), Kor (Henker) und Boron (Rabenvater).
\end{shaded*}

\end{document}